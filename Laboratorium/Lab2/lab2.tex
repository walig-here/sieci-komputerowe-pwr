\documentclass[a4paper,12pt]{article}
\usepackage{polski}
\usepackage[utf8]{inputenc}
\usepackage[OT4]{fontenc}
\usepackage{mathtools}
\usepackage{float}
\usepackage{graphicx}
\usepackage{multirow}
\usepackage{xcolor}
\usepackage{listings}

\newcommand{\h}[1]{\noindent \bf #1 \rm \\ \noindent}
\newcommand{\italic}[1]{\it #1 \rm}

\definecolor{winbackground}{HTML}{000000}
\definecolor{wintext}{HTML}{FFFFFF}

\lstset{
	language=sh,
	backgroundcolor=\color{winbackground},
	basicstyle=\ttfamily\color{wintext},
	keywordstyle=\color{white}\bfseries,
	commentstyle=\color{white},
	stringstyle=\color{red},
	frame=single,
	showstringspaces=false,
	breaklines=true,
	breakatwhitespace=true,
	tabsize=4,
}

\begin{document}

\begin{center}
	\LARGE
	Sieci Komputerowe \\
	\large
	LABORATORIUM 2 
\end{center}
\vspace{1cm}

\h{Struktura wiadomości ICMP:}
Wiadomość ICMP składa się nie tylko z przesyłanych danych ale również dodatkowych informacji nazywanych nagłówkiem. Łącznie jest to 8 dodatkowych bajtów.
\begin{table}[H]
	\centering
	\begin{tabular}{|ccc|}
		\hline
		\multicolumn{1}{|c|}{\begin{tabular}[c]{@{}c@{}}Typ\\ 1B\end{tabular}} & \multicolumn{1}{c|}{\begin{tabular}[c]{@{}c@{}}Kod\\ 1B\end{tabular}} & \begin{tabular}[c]{@{}c@{}}Suma kontrolna\\ 2B\end{tabular}  \\ \hline
		\multicolumn{2}{|c|}{\begin{tabular}[c]{@{}c@{}}Identyfikator\\ 2B\end{tabular}}                                                               & \begin{tabular}[c]{@{}c@{}}Nr. sekwencyjny\\ 2B\end{tabular} \\ \hline
		\multicolumn{3}{|c|}{\begin{tabular}[c]{@{}c@{}}Dane\\ n * 4B\end{tabular}}                                                                                                                                    \\ \hline
	\end{tabular}
\end{table}

\begin{itemize}
	\item \italic{Type} - typ wiadomości
	\item \italic{Kod} - szczegółowy rodzaj wiadomości
	\item \italic{Suma kontrolna} - suma kontrolna nagłówka i danych
	\item \italic{Identyfikator} - dodatkowe informacje do identyfikacji sesji. Młodszy bit przedstawia go w notacji big endian a młodszy w little endian
	\item \italic{Nr. sekwencyjny} - dodatkowe informacje do identyfikacji kolejności zapytań w ramach sesji. Starszy bit zawiera dane w notacji big endian a młodszy little endian.
\end{itemize}
\vspace{5mm}

\h{Narzut na wiadomość:}
Ilość dodatkowych bajtów wynikająca z danych zawartych w protokołach sieciowych, w które "opakowana" jest wiadomość. Inaczej nazywane bitami służbowymi.\\

\h{Time to live (TTL):}
Ilość skoków między routerami, jakie może wykonać pakiet, zanim zostanie usunięty z sieci. Każdy router zmniejsza TTL wiadomości o 1, aż do osiągnięcia przez nie wartości 0. W takim wypadku pakiet jest usuwany przez router.\\

\h{Metody dostępu do narzędzi konfiguracyjnych urządzeń sieciowych:}
Istnieją cztery metody dostępu do konsolowego interfejsu konfiguracyjnego urządzenia sieciowego:
\begin{itemize}
	\item \italic{Konsola} - pierwszą metodą dostępu, pozwalającą na konfigurację urządzenia sieciowego jest konsola. Łączymy się z urządzeniem za pomocą kabla wpiętego do specjalnego portu konsoli.
	
	\item \italic{SSH} - druga metoda wykorzystuje bezpieczne połączenie z interfejsem konsolowym za pomocą SSH (Secure Shell).	Tego typu połączenie wymaga połączenia urządzenia z siecią oraz odpowiedniej konfiguracji.
	
	\item \italic{Telenet} - trzecia metoda to połączenie za pomocą Telenet. Nie jest ono tak bezpieczne jak SSH. Wszystkie zapytania są w nim wysyłane w formie surowego tekstu przez sieć. Nie są szyfrowane.
	
	\item \italic{AUX} - czwarta metoda zakładająca wykorzystanie portu pomocniczego AUX. Używana jest ona w przypadku niektórych routerów. Tego typu połączenie jest wykonywane za pomocą sieci telefonicznej. Nie wymaga ona połączenia z siecią ani wcześniejszej konfiguracji.
\end{itemize}
\vspace{5mm}

\noindent
Każda z tych metoda wymaga zainstalowania oprogramowania emulującego terminal (np. PuTTY).\\

\h{Tryby poleceń w CLI:}
Urządzenia Cisco mają dwa podstawowe tryby poleceń, które w interfejsie konsolowym rozróżniamy za pomocą znaku zachęty.
\begin{table}[H]
	\centering
	\begin{tabular}{|l|l|c|}
		\hline
		\multicolumn{1}{|c|}{\textbf{Tryb}} & \multicolumn{1}{c|}{\textbf{Opis}}                                                                                              & \textbf{Znak zachęty} \\ \hline
		Użytkownika                         & \begin{tabular}[c]{@{}l@{}}Pozwala na monitorowanie sieci.\\ Nie pozwala na wykonywanie \\ zaawansowanych poleceń.\end{tabular} & \textgreater{}        \\ \hline
		Uprzywilejowany                     & Daje dostęp do wszystkich poleceń.                                                                                              & \#                    \\ \hline
	\end{tabular}
\end{table}
\vspace{5mm}

\newpage
\h{Tryby konfiguracji i tryby podrzędne:}
W celu skonfigurowania urządzenia należy przejść do trybu konfiguracji globalnej. Polecenia wykonywane w tym trybie wpływają na całe urządzenie. Znak zachęty tego trybu to:
\begin{center}
	\it \large
	[nazwa urządzenia](config)\#
\end{center}
\vspace{5mm}

\noindent
W trybie konfiguracji globalnej możemy aktywować bardziej szczegółowe tryby konfiguracji:
\begin{itemize}
	\item \italic{Linii} - konfiguracja połączeń do urządzenia takich jak SSH, Telenet itd. Znak zachęty to: 
	\begin{center}
		\it \large
		[nazwa urządzenia](config-line)\#
	\end{center}

	\item \italic{Interfejsu} - konfiguracja portu przełącznika lub interfejsu sieciowego routera. Znak zachęty to: 
	\begin{center}
		\it \large
		[nazwa urządzenia](config-if)\#
	\end{center}
\end{itemize}
\vspace{5mm}

\h{Polecenie "enable":}
Polecenie trybu użytkownika pozwalające na przejście do trybu uprzywilejowanego:
\begin{lstlisting}
Switch> enable
\end{lstlisting}
\vspace{5mm}

\h{Polecenie "disable":}
Polecenie trybu uprzywilejowanego pozwalająca na powrót do trybu użytkownika.
\begin{lstlisting}
Switch# disable
\end{lstlisting}
\vspace{5mm}

\h{Polecenie "configure terminal":}
Polecenie trybu uprzywilejowanego, pozwalająca na przejście do trybu konfiguracji globalnej.
\begin{lstlisting}
Switch# configure terminal
\end{lstlisting}
\vspace{5mm}

\newpage
\h{Polecenie "exit":}
Polecenie pozwalające na przejście "o poziom wyżej" w hierarchii trybów. Przykładowo z podtrybu konfiguracji linii do trybu konfiguracji globalnej.
\begin{lstlisting}
Switch# exit
\end{lstlisting}
\vspace{5mm}

\h{Polecenie "end":}
Polecenie pozwalające na przejście z dowolnego trybu podrzędnego bezpośrednio do trybu uprzywilejowanego. Może być zastąpione skrótem klawiszowym CTRL+Z.
Polecenie trybu uprzywilejowanego, pozwalająca na przejście do trybu konfiguracji globalnej.
\begin{lstlisting}
Switch(config-line)# end
\end{lstlisting}
\vspace{5mm}

\h{Polecenie "line":}
Polecenie pozwalające przejść do podtrybu konfiguracji linii. Polecenie wymaga parametrów takich jak typ linii oraz numer. 
\begin{lstlisting}
Switch(config)# line <typ linii> <numer>
\end{lstlisting}
\vspace{5mm}

\h{Polecenie "interface":}
Polecenie pozwalające przejść do podtrybu konfiguracji interfejsu.Polecenie wymaga parametrów takich jak typ interfejsu oraz numer.
\begin{lstlisting}
Switch(config)# interface <typ interfejsu> <numer>
\end{lstlisting}
\vspace{5mm}

\h{Polecenie "?":}
Gdy wpisane samotnie wyświetla wszystkie polecenia dostępne w aktualnie aktywnym trybie.
\begin{lstlisting}
Switch>?
\end{lstlisting}
\vspace{5mm}

\noindent
Gdy wpiszemy część komendy a następnie "?" to interfejs konsolowy poda nam wszystkie komendy, o jakie może nam chodzić. Pozwala to na przypomnienie sobie nazw komend.
\begin{lstlisting}
Switch# pi?
\end{lstlisting}
\vspace{5mm}

\noindent
Gdy wypiszemu "?" po nazwie komendy, to otrzymamy instrukcje jej użycia.
\begin{lstlisting}
Switch# ping ?
\end{lstlisting}
\vspace{5mm}

\newpage
\h{Skróty klawiszowe:}
Interfejs konsolowy zawiera wiele skrótów klawiszowych, które pozwalają na łatwiejszą nawigację i użytkowanie.
\begin{table}[H]
	\centering
	\begin{tabular}{|c|l|}
		\hline
		\textbf{Skrót} & \multicolumn{1}{c|}{\textbf{Opis}}                                                                \\ \hline
		TAB            & Uzupełnia nazwę polecenia                                                                         \\ \hline
		CTRL+SHIFT+6   & \begin{tabular}[c]{@{}l@{}}Zatrzymuje wyszukiwanie DNS, ping i \\ tracert.\end{tabular}           \\ \hline
		CTRL+Z         & \begin{tabular}[c]{@{}l@{}}Wraca z trybu konfiguracji do trybu\\ urprzywilejowanego.\end{tabular} \\ \hline
	\end{tabular}
\end{table}
\vspace{5mm}

\h{Polecenie "hostname":}
Polecenie pozwalające na ustalenie nazwy urządzenia. Jest to polecenie trybu konfiguracji globalnej. Wymaga podania nowej nazwy urządzenia jako argumentu.
\begin{lstlisting}
Switch(config)#hostname <nowa nazwa>
\end{lstlisting}
\vspace{5mm}

\h{Polecenie "no hostname":}
Polecenie pozwalające na przywrócenie domyślnej nazwy urządzenia. Jest to polecenie trybu konfiguracji globalnej.
\begin{lstlisting}
Switch(config)#no hostname
\end{lstlisting}
\vspace{5mm}

\h{Polecenie "password":}
Polecenie trybu konfiguracji linii. Pozwala na ustalenie hasła dostępu do trybu użytkownika w trakcie połączenia na aktywnej linii.
\begin{lstlisting}
Switch(confit-line)#password <haslo>
\end{lstlisting}
\vspace{5mm}

\noindent
Wszystkie ustalone w ten sposób hasła są przechowywane pliku konfiguracyjnym \it running-config\rm.\\

\h{Polecenie "login":}
Polecenie trybu konfiguracji linii. Pozwala na włączenie wymogu wpisania hasła w wypadku próby aktywacji trybu użytkownika na aktywnej linii.
\begin{lstlisting}
	Switch(confit-line)#login
\end{lstlisting}
\vspace{5mm}

\h{Polecenie "enable secret":}
Polecenie trybu konfiguracji globalnej. Pozwala na ustalenie hasła niezbędnego do aktywacji trybu uprzywilejowanego.
\begin{lstlisting}
Switch(confit)#enable secret <haslo>
\end{lstlisting}
\vspace{5mm}

\h{Szyfrowanie haseł:}
Ustalane za pomocą komendy \it password \rm hasła są zapisywane na urządzeniu za pomocą zwykłego testu. Aby zaszyfrować hasła zapisane na urządzeniu należy użyć polecenia \it service password-encryption\rm. Jest to polecenie konfiguracji globalnej.
\begin{lstlisting}
Switch(config)#service password-encryption
\end{lstlisting}
\vspace{5mm}

\h{Polecenie "show":}
Polecenie trybu uprzywilejowanego. Pozwala na wyświetlenie wskazanego elementu.
\begin{lstlisting}
Switch#show <element>
\end{lstlisting}
\vspace{5mm}

\h{Polecenie "banner motd":}
Polecenie trybu konfiguracji globalnej. Pozwala na ustalenie wiadomości wyświetlanej przy próbie zalogowania się do konsoli urządzenia.
\begin{lstlisting}
Switch(config)#banner motd #<wiadomosc>#
\end{lstlisting}
\vspace{5mm}

\h{Plik running-config:}
Plik zawierający aktualnie zapisaną w pamięci RAM konfigurację urządzenia. Zostanie on zresetowany, gdy urządzenie zostanie odłączone od zasilania. To właśnie do running-config zapisują się wprowadzone za pośrednictwem konsoli zmiany.\\

\h{Plik startup-config:}
Plik zawierający zapisaną w pamięci NVRAM konfigurację urządzenia. Zostanie on w nienaruszonym stanie przy odłączeniu urządzenia od zasilania.\\

\newpage
\h{Zapisanie aktualnej konfiguracji:}
Aby zapisać aktualną konfigurację zawartą w running-config do pliku startup-config należy użyć polecenia \it copy\rm. Można go użyć w trybie uprzywilejowanym.
\begin{lstlisting}
Switch#copy running-config startup-config
\end{lstlisting}
\vspace{5mm}

\h{Polecenie "reload":}
Polecenie trybu uprzywilejowanego. Pozwala zresetować konfigurację do stanu zapisanego w pliku startup-config. Minusem tego polecenia jest to, ze powoduje ono chwilowe przejście urządzenia w tryb offline.
\begin{lstlisting}
Switch#reload
\end{lstlisting}
\vspace{5mm}

\h{Usunięcie zapisanej konfiguracji:}
Polecenie \it erase startup-config \rm pozwala na pełne wyczyszczenie zapisanej w pliku startup-config konfiguracji urządzenia.
\begin{lstlisting}
Switch#erase startup-config
\end{lstlisting}
\vspace{5mm}

\h{Maska podsieci:}
Maska podsieci IPv4 to 32-bitowa wartość, która odróżnia część sieciową adresu od części hosta. Maska w połączeniu z właściwym adresem IP określa do której podsieci należy urządzenie.\\

\h{Adres bramy domyślnej:}
Adres IP routera, którego urządzenie będzie używało, aby uzyskać dostęp do sieci zadanych (także Internetu).\\

\h{Protokół DHCP:}
Protokół automatycznego przypisania i konfiguracji adresów IPv4. Jedna z najpowszechniej używanych w sieci technologii.\\

\h{Interfejs wirtualny SVI:}
Interfejs obecny w oprogramowaniu przełącznika, który pozwala na zdalne zarządzanie urządzenia przy użyciu protokołów IP.\\

\newpage
\h{Polecenie "ip address":}
Polecenie używane do skonfigurowania adresu IP oraz maski podsieci urządzenia na porcie SVI. Jest to polecenie trybu konfiguracji interfejsu \it vlan 1\rm.
\begin{lstlisting}
Switch(config-if)#ip address <adres> <maska>
\end{lstlisting}
\vspace{5mm}

\h{Polecenie "ip defaulf-gateway":}
Polecenie trybu konfiguracji interfejsu. Pozwala na ustalenie bramy domyślnej dla danego urządzenia.
\begin{lstlisting}
Switch(config-if)#ip default-gateway <adres>
\end{lstlisting}
\vspace{5mm}

\h{Polecenie "no shutdown":}
Polecenie trybu konfiguracji interfejsu. Pozwala ona na aktywacje wirtualnego interfejsu (np. SVI).
\begin{lstlisting}
	Switch(config-if)#no shutdown
\end{lstlisting}
\vspace{5mm}

\h{Wyświetlanie danych o wszystkich portach:}
Poleceniem służącym do wyświetlenia wszystkich danych o portach danego urządzenia jest \it show ip interface brief\rm. Jest to polecenie trybu uprzywilejowanego.
\begin{lstlisting}
Switch#show ip interface brief
\end{lstlisting}
\vspace{5mm}

\end{document}