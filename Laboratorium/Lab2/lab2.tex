\documentclass[a4paper,12pt]{article}
\usepackage{polski}
\usepackage[utf8]{inputenc}
\usepackage[OT4]{fontenc}
\usepackage{mathtools}
\usepackage{float}
\usepackage{graphicx}
\usepackage{multirow}

\newcommand{\h}[1]{\noindent \bf #1 \rm \\ \noindent}
\newcommand{\italic}[1]{\it #1 \rm}

\begin{document}

\begin{center}
	\LARGE
	Sieci Komputerowe \\
	\large
	LABORATORIUM 2 
\end{center}
\vspace{1cm}

\h{Struktura wiadomości ICMP:}
Wiadomość ICMP składa się nie tylko z przesyłanych danych ale również dodatkowych informacji nazywanych nagłówkiem. Łącznie jest to 8 dodatkowych bajtów.
\begin{table}[H]
	\centering
	\begin{tabular}{|ccc|}
		\hline
		\multicolumn{1}{|c|}{\begin{tabular}[c]{@{}c@{}}Typ\\ 1B\end{tabular}} & \multicolumn{1}{c|}{\begin{tabular}[c]{@{}c@{}}Kod\\ 1B\end{tabular}} & \begin{tabular}[c]{@{}c@{}}Suma kontrolna\\ 2B\end{tabular}  \\ \hline
		\multicolumn{2}{|c|}{\begin{tabular}[c]{@{}c@{}}Identyfikator\\ 2B\end{tabular}}                                                               & \begin{tabular}[c]{@{}c@{}}Nr. sekwencyjny\\ 2B\end{tabular} \\ \hline
		\multicolumn{3}{|c|}{\begin{tabular}[c]{@{}c@{}}Dane\\ n * 4B\end{tabular}}                                                                                                                                    \\ \hline
	\end{tabular}
\end{table}

\begin{itemize}
	\item \italic{Type} - typ wiadomości
	\item \italic{Kod} - szczegółowy rodzaj wiadomości
	\item \italic{Suma kontrolna} - suma kontrolna nagłówka i danych
	\item \italic{Identyfikator} - dodatkowe informacje do identyfikacji sesji. Młodszy bit przedstawia go w notacji big endian a młodszy w little endian
	\item \italic{Nr. sekwencyjny} - dodatkowe informacje do identyfikacji kolejności zapytań w ramach sesji. Starszy bit zawiera dane w notacji big endian a młodszy little endian.
\end{itemize}
\vspace{5mm}

\h{Narzut na wiadomość:}
Ilość dodatkowych bajtów wynikająca z danych zawartych w protokołach sieciowych, w które "opakowana" jest wiadomość. Inaczej nazywane bitami służbowymi.\\

\h{Time to live (TTL):}
Ilość skoków między routerami, jakie może wykonać pakiet, zanim zostanie usunięty z sieci. Każdy router zmniejsza TTL wiadomości o 1, aż do osiągnięcia przez nie wartości 0. W takim wypadku pakiet jest usuwany przez router.

\end{document}